%%%%%%%%%%%%%%%%%%%%%%%%%%%%%%%%%%%%%%%%%
% Frequently Asked Questions
% LaTeX Template
% Version 1.0 (22/7/13)
%
% This template has been downloaded from:
% http://www.LaTeXTemplates.com
%
% Original author:
% Adam Glesser (adamglesser@gmail.com)
%
% License:
% CC BY-NC-SA 3.0 (http://creativecommons.org/licenses/by-nc-sa/3.0/)
%
%%%%%%%%%%%%%%%%%%%%%%%%%%%%%%%%%%%%%%%%%

\documentclass[12pt]{article}
\usepackage{minted}
\usepackage[dvipsnames]{xcolor}
\usepackage[margin=1in]{geometry} % Required to make the margins smaller to fit more content on each page
\usepackage[linkcolor=BlueViolet]{hyperref} % Required to create hyperlinks to questions from elsewhere in the document
\hypersetup{pdfborder={0 0 0}, colorlinks=true, urlcolor=blue} % Specify a color for hyperlinks
\usepackage{todonotes} % Required for the boxes that questions appear in
\usepackage{tocloft} % Required to give customize the table of contents to display questions
\usepackage{microtype} % Slightly tweak font spacing for aesthetics

\usepackage{fontspec}
\usepackage{unicode-math}
\RequirePackage{fontspec}
\RequirePackage{unicode-math}
\setmainfont[
Path = fonts/,
ItalicFont=NewCMMono10-BookItalic.otf,
BoldFont=NewCMMono10-Bold.otf,%
BoldItalicFont=NewCMMono10-BoldOblique.otf,%
SmallCapsFeatures={Numbers=OldStyle}]{NewCMMono10-Book.otf}

\setmonofont[
Path=fonts/,
Extension=.ttf]{FiraCode-Light.ttf}

\setlength\parindent{0pt} % Removes all indentation from paragraphs

% Create and define the list of questions
\newlistof{questions}{faq}{\large At a glance} % This creates a new table of contents-like environment that will output a file with extension .faq
\setlength\cftbeforefaqtitleskip{4em} % Adjusts the vertical space between the title and subtitle
\setlength\cftafterfaqtitleskip{1em} % Adjusts the vertical space between the subtitle and the first question
\setlength\cftparskip{.3em} % Adjusts the vertical space between questions in the list of questions

% Create the command used for questions
\newcommand{\question}[1] % This is what you will use to create a new question
{
\refstepcounter{questions} % Increases the questions counter, this can be referenced anywhere with \thequestions
\par\noindent % Creates a new unindented paragraph
\phantomsection % Needed for hyperref compatibility with the \addcontensline command
\addcontentsline{faq}{questions}{#1} % Adds the question to the list of questions
\todo[inline, color=RoyalBlue!20]{\textbf{#1}} % Uses the todonotes package to create a fancy box to put the question
\vspace{1em} % White space after the question before the start of the answer
}

% Uncomment the line below to get rid of the trailing dots in the table of contents
%\renewcommand{\cftdot}{}

% Uncomment the two lines below to get rid of the numbers in the table of contents
%\let\Contentsline\contentsline
%\renewcommand\contentsline[3]{\Contentsline{#1}{#2}{}}

\begin{document}
\definecolor{codeBackground}{HTML}{240237}

%----------------------------------------------------------------------------------------
%	TITLE AND LIST OF QUESTIONS
%----------------------------------------------------------------------------------------

\begin{center}
\Huge{\bf Docker Tips, Tricks, and Notes} % Main title
\end{center}

\listofquestions % This prints the subtitle and a list of all of your questions

\newpage % Comment this if you would like your questions and answers to start immediately after table of questions

%----------------------------------------------------------------------------------------
%	QUESTIONS AND ANSWERS
%----------------------------------------------------------------------------------------

\question{What would be a quick start for a containerised app}

Follow a 3-step process:
\begin{enumerate}
\item Define a \texttt{Dockerfile} for your app's environment so it can be replicated anywhere.
\item Define a Docker Compose file for the services needed in your app.
\item Run the following command in the working directory where these above mentioned files are located:
\begin{minted}
[
frame=lines,
framesep=2mm,
baselinestretch=1.2,
bgcolor=codeBackground!30,
fontsize=\footnotesize,
]
{shell}
docker compose up
\end{minted}
\end{enumerate}

%------------------------------------------------

\question{What is Docker Compose?}

Docker Compose is a tool for running multi-container applications on Docker defined using the Compose file format. 

%------------------------------------------------

\question{What is a Compose file?}

The Compose file is a YAML file that defines services, networks, volumes, configurations and secrets. The default name of the Compose file can be \texttt{compose.yaml}, \texttt{compose.yml}, \texttt{docker-compose.yaml}, or \texttt{docker-compose.yml}.

%------------------------------------------------

%----------------------------------------------------------------------------------------

\end{document}